%-------------------------------------------------------------------------------
% rtl66
%-------------------------------------------------------------------------------
%
% \file        rtl66
% \library     Documents
% \author      Chris Ahlstrom
% \date        2022-07-12
% \update      2022-07-12
% \version     $Revision$
% \license     $XPC_GPL_LICENSE$
%
%     This document provides LaTeX documentation for the Rtl66 library.
%
%-------------------------------------------------------------------------------

\section{Walk-through}
\label{sec:walkthrough}

   \index{ALSA}
   Let's start by walking through one of the test applications,
   \texttt{tests/qmidiin}.  We run our debugger, set a breakpoint
   where the input object is created, and run using ALSA.

   \begin{verbatim}
      $ cgdb build/tests/qmidiin
      (gdb) r --alsa
   \end{verbatim}

   Here is what happens:

   \begin{enumerate}
      \item \textbf{Select ALSA as the API}.
         \begin{itemize}
            \item Use the command-line to set the \texttt{--alsa} option.
            \item \texttt{rt\_simple\_cli()} sets the API to
               \texttt{rtl::rtmidi::api::alsa}.
            \item Retrieved the API with \texttt{rtl::rtmidi::desired\_api()}.
         \end{itemize}
      \item\textbf{Create the MIDI input proxy}.
         Pass the API to the
         \texttt{rtl::rtmidi\_in} constructor.
         The three possible parameters are:
         \begin{itemize}
            \item The API.  For \textsl{Linux}, they are
               \textsl{JACK}, \textsl{ALSA}, and,
               later, \textsl{Pipewire}. The default is "unspecified",
               which would invoke the fallback sequence that
               detects which APIs are available (currently
               JACK to ALSA).
            \item The client-name. Empty by default. For testing,
               the \texttt{--client name}.
            \item The input-queue size. Defaults to 100 MIDI messages.
         \end{itemize}
      \item\textbf{Open the selected API}.
         \texttt{rtl::rtmidi\_in::open\_midi\_api()} detects the compiled
         APIs and tries to create the desired one.
      \item\textbf{Create the actual MIDI input}.
         \texttt{rtl::midi\_alsa\_in()} with default client-name "rtl66 in".
      \item\textbf{Set ALSA data structure to defaults}.
         Initializes \texttt{rtl::midi\_alsa\_data} to useless defaults.
      \item\textbf{Set up the ALSA client}.
         The call sequence is:
         \begin{itemize}
            \item \texttt{rtl::midi\_alsa\_in::initialize()}
            \item \texttt{rtl::midi\_alsa::impl\_initialize()}
         \end{itemize}
            The ALSA sequencer handle (\texttt{snd\_seq\_t}) gets set up.
            Also called the "client" handle, not to be confused with
            "port" handles.
      \item\textbf{Set the tempo and PPQN}.
         \texttt{rtl::midi\_alsa::impl\_set\_tempo()}.
         Sets the default tempo and PPQN.
      \item \textbf{Get the number of ports}.
         The call sequence is:
         \begin{itemize}
            \item \texttt{rtmidi\_in::get\_port\_count()} calls
            \item Get the \texttt{midi\_api} pointer returned by
               \texttt{rtl::rtmidi::rt\_api\_ptr()}.
            \item Calll \texttt{midi\_alsa\_in::get\_port\_count()}
               via this pointer.
            \item \texttt{rtl::midi\_alsa::impl\_get\_port\_count()}
               for MIDI input.
            \item This function also gets the port capabilities via ALSA.
            \item Call the static function \texttt{rtl::get\_port\_info()}
               (in \texttt{midi\_alsa.cpp})
               with a port value of \texttt{-1}, which indicates "all ports".
         \end{itemize}
      \item \textbf{Open the input port}.
         The call sequence is:
         \begin{itemize}
            \item \texttt{rtl::rtmidi\_in::open\_port()}.
            \item \texttt{rtl::midi\_alsa\_in::open\_port()}.
            \item \texttt{rtl::midi\_alsa::impl\_open\_port()}.
            \item \texttt{rtl::get\_port\_info()}, this time for port 0.
         \end{itemize}
      \item \textbf{Get port information}.
         \texttt{rtl::get\_port\_info()} iterates through all input ports,
         retrieving port information (wasteful!) until it gets the 
         proper kind of port where the index of the port matches the
         desired port number.
         \texttt{}.
      \item \textbf{Create the port}.
         \texttt{rtl::rtmidi\_in::open\_port()} then gets the sender and
         receiver addresses, sets the client, the port, the capabilities, type,
         and number of channels in the port (16), and the port's name.
         (EMPTY!!!!!!)
         Optionally sets time-stamping information.
      \item \textbf{Port subscription}.
         \texttt{midi\_alsa::impl\_subscription()} is called to set up
         ALSA port subscription, a fancy way to connect ports.
   \end{enumerate}

   If ALSA were not specified, then
   \texttt{rtl::rtmidi::fallback\_api()} would be
   called to try JACK first, then ALSA.
   (Later, PipeWire will be the first choice).

   Also note that of the \texttt{midi\_api} functions call
   an "impl" function as a helper, to reduce code duplication.


\subsection{Supported APIs}
\label{subsec:supported_apis}

   \textsl{Rtl66} supports the following MIDI APIs:

   \begin{itemize}
      \item PipeWire. (Planned, not yet implemented).
      \item JACK.
      \item ALSA.
      \item Windows Multi-Media.
      \item Mac OSX Core.
      \item Web MIDI.
      \item Dummy.
   \end{itemize}

\subsection{xxxx}
\label{subsec:xxxx}

%-------------------------------------------------------------------------------
% vim: ts=3 sw=3 et ft=tex
%-------------------------------------------------------------------------------
