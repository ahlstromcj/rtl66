%-------------------------------------------------------------------------------
% midi_meta_and_sysex
%-------------------------------------------------------------------------------
%
% \file        midi_meta_and_sysex.tex
% \library     Documents
% \author      Chris Ahlstrom
% \date        2024-05-13
% \update      2024-05-13
% \version     $Revision$
% \license     $XPC_GPL_LICENSE$
%
%     Provides a discussion of the MIDI meta and sysex message formats
%     and how they are handled,
%
%-------------------------------------------------------------------------------

\section{MIDI Meta and System Exclusive Events}
\label{sec:midi_and_midi_notes}

   This section describes how the \textsl{Rtl66} library handles MIDI
   Meta and SysEx events.  There are many intricacies to work with.
   It also covers, concisely, MIDI operations not yet implemented, but
   important to keep in the "to do" queue.

   Also helpful is additional information about other MIDI events in the
   \textsl{Seq66} manual, section \textsl{MIDI Formats}
   \cite{seq66}.

\subsection{MIDI Meta Events}
\label{subsec:midi_meta_events}

   Meta events are non-MIDI data of various sorts consisting of a fixed prefix,
   an event type, a length field, and the meta event data.
   Meta events are \textsl{never} sent to a device.
   \textsl{Rtl66} tries to load, store, and write all meta events, even
   where it cannot use the meta data.
 
   \texttt{0xFF + meta\_type + v\_length + event\_data\_bytes}

   \begin{itemize}
      \item \texttt{meta\_type} is 1 byte, expressing one of the meta event
         types shown below.
      \item \texttt{v\_length} is length of meta event data, a variable
         length ("varinum") value.
      \item \texttt{event\_data\_bytes} is the actual event data.
   \end{itemize}

   Here, we summarize the MIDI meta events data.
   Two-character names represent a single byte; \textsl{len}
   represents a variable-length value.

   \begin{enumber}
      \item \texttt{FF 00 02 ss ss}.
         Sequence Number.
      \item \texttt{FF 01 len text}.
         Text Event.
      \item \texttt{FF 02 len text}.
         Copyright Notice.
      \item \texttt{FF 03 len text}.
         Sequence/Track Name.
      \item \texttt{FF 04 len text}.
         Instrument Name.
      \item \texttt{FF 05 len text}.
         Lyric.
      \item \texttt{FF 06 len text}.
         Marker.
      \item \texttt{FF 07 len text}.
         Cue Point.
      \item \texttt{FF 08 len text}.
         Program Name.
      \item \texttt{FF 09 len text}.
         Port Name.
%     \item \texttt{FF 0A through 0F len text}.
%        Other kinds of text events.
      \item \texttt{FF 2F 00}.
         End of Track.
      \item \texttt{FF 51 03 tt tt tt}.
         Set Tempo, us/qn.
      \item \texttt{FF 54 05 hr mn se fr ff}.
         SMPTE Offset.
      \item \texttt{FF 58 04 nn dd cc bb}.
         Time Signature.
      \item \texttt{FF 59 02 sf mi}.
         Key Signature.
      \item \texttt{FF 7F len data}.
         Sequencer-Specific.
%     \item \texttt{FF F0 len data F7}.  FIXME in Seq66 manual!!!
%        System-Exclusive
   \end{enumber}

   We need to make sure we read, save, and restore all of the items above,
   even if \textsl{Rtl66} doesn't use them.
   Some are deprecated in the MIDI standard.
   The next sections describe the events that \textsl{Rtl66} tries to
   handle.  These are:

   \begin{itemize}
      \item Generic Meta Messages.
      \item Sequence Number (0x00).
      \item Track Name (0x03).
      \item Other text events(0x04 to 0x09)..
      \item End-of-Track (0x2F).
      \item Set Tempo (0x51).
      \item Time Signature (0x58).
      \item Key Signature (0x59).
      \item Sequencer-Specific (0x7F) (handled differently in Rtl66).
      \item System Exclusive (0xF0).
   \end{itemize}

   Also, all the text events should be handled by \textsl{Rtl66}, but
   a lot more testing is needed.

\subsubsection{Generic Meta Messages}
\label{subsubsec:midi_meta_generic}

   \begin{verbatim}
      FF nn len data...
   \end{verbatim}

   Any meta message not explicitly handled is read and is stored in a track.
   Nothing else is done with it.
   It can be written back out to a MIDI file as is.

\subsubsection{Track/Sequence Number (0x00)}
\label{subsubsec:midi_meta_sequence_number}

   \begin{verbatim}
      FF 00 02 ss ss
      FF 00 00
   \end{verbatim}

   This optional event must occur at the beginning of a track,
   before any non-zero delta-times, and before any transmittable MIDI
   events.  It specifies the number of a sequence. The second form,
   lacking the "ss ss" (sequence number bytes), the track's location in the
   file is use. \textsl{Rtl66} will read that form, but convert it to
   the first form.

\subsubsection{Track/Sequence Name (0x03)}
\label{subsubsec:midi_meta_sequence_name}

   \begin{verbatim}
      FF 03 len text
   \end{verbatim}

   If in a format 0 track, or the first track in a format 1 file, the name
   of the sequence (song).  Otherwise, the name of the track.

\subsubsection{End of Track/Sequence (0x2F)}
\label{subsubsec:midi_meta_end_of_track}

   \begin{verbatim}
      FF 2F 00
   \end{verbatim}

   This event is \textsl{not optional}.
   It is included so that an exact ending
   point may be specified for the track, so that it has an exact length,
   which is necessary for tracks which are looped or concatenated.

\subsubsection{Set Tempo Event (0x51)}
\label{subsubsec:midi_meta_set_tempo}

   The MIDI Set Tempo meta event sets the tempo of a MIDI sequence in terms of
   the microseconds per quarter note.  This is a meta message, so this event is
   never sent over MIDI ports to a MIDI device.
   After the delta time, this event consists of six bytes, with three
   tempo bytes.

   \begin{verbatim}
      FF 51 03 tt tt tt       Example: FF 51 03 07 A1 20
   \end{verbatim}

   \begin{enumber}
      \item 0xFF is the status byte that indicates this is a Meta event.
      \item 0x51 the meta event type that signifies this is a Set Tempo event.
      \item 0x03 is the length of the event, always 3 bytes.
      \item The remaining three bytes carry the number of microseconds per
         quarter note.  For example, the three bytes above form the hexadecimal
         value 0x07A120 (500000 decimal), which means that there are 500,000
         microseconds per quarter note.
   \end{enumber}

   Since there are 60,000,000 microseconds per minute, the event above
   translates to: set the tempo to 60,000,000 / 500,000 = 120 quarter notes per
   minute (120 beats per minute).

   This event normally appears in the first track. If not, the default tempo is
   120 beats per minute.  This event is important if the MIDI time division is
   specified in "pulses per quarter note", which does not itself define the
   length of the quarter note. The length of the quarter note is then
   determined by the Set Tempo meta event.

   Representing tempos as time per beat instead of beat per time allows
   absolutely exact synchronization with a time-based sync protocol
   such as SMPTE time code or MIDI time code.  This amount of accuracy
   in the tempo resolution allows a four-minute piece at 120 beats per minute
   to be accurate within 500 usec at the end of the piece.

\subsubsection{Time Signature Event (0x58)}
\label{subsubsec:midi_meta_time_sig}

   After the delta time, this event consists of seven bytes of data:

   \begin{verbatim}
      FF 58 04 nn dd cc bb
   \end{verbatim}

   The time signature is expressed as four numbers.
   \texttt{nn} and \texttt{dd} represent the numerator and denominator of the
   time signature as it would be notated.  The denominator is a negative power
   of two:  2 represents a quarter-note, 3 represents an eighth-note, etc.  The
   \texttt{cc} parameter expresses the number of MIDI clocks in a metronome
   click.  The \texttt{bb} parameter expresses the number of notated 32nd-notes
   in a MIDI quarter-note (24 MIDI Clocks).
   Example:

   \begin{verbatim}
      FF 58 04 04 02 18 08
   \end{verbatim}

   \begin{enumber}
      \item 0xFF is the status byte that indicates this is a Meta event.
      \item 0x58 the meta event type that signifies this is a Time Signature
         event.
      \item 0x04 is the length of the event, always 4 bytes.
      \item 0x04 is the numerator of the time signature, and ranges from 0x00
         to 0xFF.
      \item 0x02 is the log base 2 of the denominator, and is the power to
         which 2 must be raised to get the denominator.  Here, the denominator
         is 2 to 0x02, or 4, so the time signature is 4/4.
      \item 0x18 is the metronome pulse in terms of the number of
         MIDI clock ticks per click.  Assuming 24 MIDI clocks per quarter note,
         the value here (0x18 = 24) indidicates that the metronome will tick
         every 24/24 quarter note.  If the value of the sixth byte were 0x30 =
         48, the metronome clicks every two quarter notes, i.e. every
         half-note.
      \item 0x08 defines the number of 32nd notes per beat. This byte is
         usually 8 as there is usually one quarter note per beat, and one
         quarter note contains eight 32nd notes.
   \end{enumber}

   If a time signature event is not present in a MIDI sequence, a 4/4 signature
   is assumed.

% FIXME: Add this section to the Seq66 manual.

\subsubsection{Key Signature Event (0x59)}
\label{subsubsec:midi_meta_key_sig}

   After the delta time, this event consists of five bytes of data:

   \begin{verbatim}
      FF 59 02 ss kk
   \end{verbatim}

   \begin{enumber}
      \item 0xFF is the status byte that indicates this is a Meta event.
      \item 0x59 the meta event type that signifies this is a Key Signature
         event.
      \item 0x042is the length of the event, always 2 bytes.
      \item ss is the number of the scale.
      \item kk is 0 for a major scale, and 1 for a minor scale.
   \end{enumber}
   
% FIXME elaborate more on this event.

% FIXME: Add this section to the Seq66 manual.

\subsubsection{Text Events}
\label{subsubsec:midi_meta_text_events}

   Text events store an arbitrary number of bytes to represent a
   human-readable string.
   The characters should be limited to the ASCII character set.

   \begin{verbatim}
      FF nn len text
   \end{verbatim}

   Here are the values of \textsl{nn} and the text they represent:

   \begin{enumber}
      \item 0x01.
         Text Event. \texttt{FF 01 len text}.
         Supplies an arbitrary text string tagged to the track and time.
         It can be used for textual information which doesn't fall into
         one of the more specific categories below.
      \item 0x02.
         Copyright. \texttt{FF 02 len text}.
         The text specifies copyright information for the sequence (song).
         This is usually placed at time 0 of the first track in the
         sequence (song).
      \item 0x03.
         Track Name. \texttt{FF 03 len text}.
         See \sectionref{subsubsec:midi_meta_sequence_name}.
      \item 0x04.
         Instrument name. \texttt{FF 04 len text}.
         The text names the instrument intended to play the contents of
         this track. Usually placed at time 0 of the track.
         Note that this meta-event is simply a description; MIDI
         synthesisers are not required (and rarely if ever) respond to it.
         This meta event is particularly useful in sequences prepared
         for synthesisers which do not conform to the General MIDI patch set,
         as it documents the intended instrument for the track when the
         sequence is used on a synthesiser with a different patch set.
      \item 0x05.
         Lyric.  \texttt{FF 05 len text}.
         The text gives a lyric intended to be sung at the given time.
         Lyrics are often broken down into separate syllables to time-align
         them more precisely with the sequence. See karoke formats.
      \item 0x06.
         Marker. \texttt{FF 06 len text}.
         The text marks a point in the sequence which occurs at the
         given time; for example "Third Movement".
      \item 0x07.
         Cue Point. \texttt{FF 07 len text}.
         The text identifies a synchronization point which occurs
         at the specified time; for example, "Door slams".
      \item 0x08.
         Program Name. \texttt{FF 08 len text}.
         The name of the program (i.e. patch) used to play the track.
         This may be different than the Sequence/Track Name.
         For example, if the name of your track is "Butterfly",
         and the track is played upon an electric piano patch,
         one may include a Program Name of "Electric Piano".
      \item 0x09.
         Port Name. \texttt{FF 09 len text}.
         The name of the MIDI device (port) where the track is routed. It
         replaces the "MIDI Port" meta event which some sequencers formally
         used to route MIDI tracks to various MIDI ports (to support
         more than 16 MIDI channels). Example: assume a
         MIDI interface that has 4 MIDI output ports. They are listed as
         "MIDI Out 1", "MIDI Out 2", "MIDI Out 3", and "MIDI Out 4". If you
         wished a particular track to use "MIDI Out 1" then you would put a
         Port Name Meta-event at the beginning of the track, with "MIDI Out
         1" as the text. All MIDI events that occur in the track, after a
         given Port Name event, will be routed to that port. In a format 0
         MIDI file, it would be permissible to have numerous Port Name events
         intermixed with MIDI events, so that the one track could address
         numerous ports. But that would likely make the MIDI file much larger
         than it need be. The Port Name event is useful primarily in format 1
         MIDI files, where each track gets routed to one particular port.
%     \item 0xXX to 0xZZ.
%        Unspecified text events. \texttt{FF 01 len text}.
   \end{enumber}

\subsection{Sequencer-Specific Meta-Events Format}
\label{subsec:midi_meta_format}

   This data, also known as
   \texttt{SeqSpec} data, provides a way to encode information
   that a specific sequencer application needs, while marking it so that other
   sequences can safely ignore the information.

   \begin{verbatim}
      FF 7F len id data
   \end{verbatim}

   The first byte of the message is the "manufacturer ID".


\subsubsection{Meta Events Handling}
\label{subsubsec:midi_meta_events_revisited}

   If the event status masks off to 0xF0 (0xF0 to 0xFF), then it is a 
   real-time event.
   If the Meta event byte is 0xF7, it is called a "Sequencer-specific",
   or "SeqSpec" event.  For this kind of event, then one to three manufacturer
   ID bytes and the length of the event are read.

   \begin{verbatim}
      Meta type:     varies         1 byte
      Meta length:   varies         1 or more bytes
   \end{verbatim}

   If the type of the
   \texttt{SeqSpec} (0xFF) meta event is 0x7F, parsing checks to see
   if it is one of the \textsl{Rtl66}-specific events.  These events are tagged
   with various values that mask off to 0x242400nn.  The parser reads the
   tag:

   \begin{verbatim}
      Prop tag:     0x242400nn      4 bytes
   \end{verbatim}

   These tags provide a way to save and recover \textsl{Seq24/Rtl66} properties
   from the MIDI file: MIDI buss, MIDI channel, time signature, sequence
   triggers, and the key, scale, and background sequence to use with the
   track/sequence.  Any leftover data for the tagged event is let go.
   Unknown tags are skipped.

   If the type of the
   \texttt{SeqSpec} (0xFF) meta event is 0x2F, then it is the
   End-of-Track marker.  The current time marks the length (in MIDI pulses) of
   the sequence.  Parsing is done for that track.

   If the type of the
   \texttt{SeqSpec} (0xFF) meta event is 0x03, then it is the
   sequence name.  The "length" number of bytes are read, and loaded as the
   sequence name.

   If the type of the
   \texttt{SeqSpec} (0xFF) meta event is 0x00, then it is the
   sequence number, which is read:

   \begin{verbatim}
      Seq number:    varies         2 bytes
   \end{verbatim}

   Note that the sequence number might be modified latter to account for the
   current \textsl{Seq24} screenset in force for a file import operation.
   Any other \texttt{SeqSpec} type is simply skipped by reading the "length"
   number of bytes.
   The remaining sections simply describe MIDI meta events in more detail, for
   reference.

\subsection{System Exclusive Event (0xF0)}
\label{subsec:midi_meta_sysex_event}

   System messages are used to send data such as patch parameters, sampler
   data, or a sequencer memory bulk dump.
   If the meta event status value is 0xF0, it is a "System-exclusive",
   or "SysEx" event.
   \textsl{Rtl66} stores these messages, but there are complications
   to be dealt with.

   SysEx messages as sent and received are of the format:

   \begin{verbatim}
     F0 ID sysex data bytes F7
   \end{verbatim}

   The ID is a manufacturer's ID. Normally a single byte, additional IDs can be
   represented by the sequence \texttt{00 xx yy}, to allow 16384 additional
   manufacturer IDs. Here is the distribution of ID numbers:

   \begin{verbatim}
               American  European   Japanese    Other       Special
   1 byte ID:  01 - 1F   20 - 3F    40 - 5F     60 - 7C     7D - 7F
   3 byte ID: 00 00 01   00 20 00   00 40 00    00 60 00
              00 1F 7F   00 3F 7F   00 5F 7F    00 7F 7F
   \end{verbatim}

   "00" and "00 00 00" are not to be used.
   Currently \textsl{Rtl66} does not try to interpret the manufacturer ID.

   There are three additional special ID codes that are \textsl{not}
   manufacturer-specific:

   \begin{itemize}
      \item \textbf{0x7D}.
         Used for private non-commercial purposes only.
         It has no standard format.
      \item \textbf{0x7E}.
         Represents a non-realtime univerals system exclusive message.
         All MIDI devices can respond to it, although not immediately.
      \item \textbf{0x7F}.
         Represents a realtime univerals system exclusive message.
         All MIDI devices can respond to it, immediately.
   \end{itemize}

   The format for 0x7E and 0x7F is the same:

   \begin{verbatim}
      F0H ID_number device_ID sub-ID#1 sub-ID#2 ... F7H
   \end{verbatim}

   An ID number of 0x7F (the 'all call' ID) indicates all devices should
   respond. As for device ID, some devices might have multiple IDs.

   For more information on these events, see
   \sectionref{subsubsec:midi_meta_universalsysex_event}.

   Now, when encoded for storage in a MIDI file, the format of the bare message
   (first line) is padded so that it starts with a delta time, and includes the
   length (a variable-length quantity) of the SysEx including the final F7, but
   excluding the F0 and the length bytes.:

   \begin{verbatim}
     F0 ID sysex data bytes F7               (message)
     delta F0 len ID sysex data bytes F7     (encoded)
   \end{verbatim}

   SysEx events can be encoded in the following ways:

   \begin{itemize}
      \item \textbf{Single SysEx Message}.
      \item \textbf{Continuation Events}.
      \item \textbf{Escape Sequence}.
   \end{itemize}

\paragraph{Single SysEx Message}
\label{paragraph:patterns_single_sysex_message}

   A SysEx message that is \textsl{sent to a device} is not quite the 
   same as a SysEx message that is \textsl{encoded in a MIDI file}.
   SysEx messages encoded in a MIDI file
   are preceded by a variable-length delta time, the byte F0,
   a variable-length length value, and the message, and a terminating F7,
   which \textsl{must} be present, and is counted in the length value.
   In the following example, the first line is the actual SysEx message
   (a General MIDI Enable message),
   and the second line is its encoding in a MIDI file, ignoring the preceding
   delta time:

   \begin{verbatim}
      F0 7E 00 09 01 F7          (message)
      F0 05 7E 00 09 01 F7       (encoded)
   \end{verbatim}

\paragraph{Continuation Events}
\label{paragraph:patterns_continuation_events}

   Some equipment needs the SysEx to be split into smaller chunks for
   processing.
   This could be accomplished with a number of smaller single-SysEx messages,
   but some manufacturers treat SysEx as if it supported running status (it
   does not).
   So the first message starts with F0, the next messages have only the SysEx
   data, and the last message ends with F7.
   However, when encoding in the MIDI file, each sub-packet begins with
   an F7.
   In between are the delta times to use to delay the sub-packets when sending
   to a slow device.
   An unencoded message with 3 packets is shown,
   with a bar for a separator for this discussion:

   \begin{verbatim}
      F0 43 12 00 | 43 12 00 43 12 00 | 43 12 00 F7
   \end{verbatim}

   Encoded with a 200-tick delta time (\texttt{81 48}) between each message,
   the F7 being used as a \textsl{continuation} byte, with length values:

   \begin{verbatim}
      00 F0 03 43 12 00
      81 48 F7 06 43 12 00 43 12 00
      81 48 F7 04 43 12 00 F7
   \end{verbatim}

   Since the first message is not terminated by F7 within the specified
   length, the next two F7s indicate a continuation.
   Note the F7 at the beginning of packets after the first one, and
   the F7 at the end of the last one.

\paragraph{Escape Sequence}
\label{paragraph:patterns_escape_sequence}

   An escape sequence is not SysEx, but it does use the F7 byte.
   It is used for encoding arbitrary bytes for messages such as Song Select.
   The first line shows such a message, and the second how it would be encoded.

   \begin{verbatim}
      F3 01
      F7 02 F3 01
   \end{verbatim}

   The format is \texttt{F7 len <all bytes to be transmitted>}.

   So how to figure out what F7 means?
   Its interpretation is as follows:

   \begin{itemize}
      \item An F0 without a terminal F7 within the specified length
         is a Casio-style multi-packet message,
         and a flag (call it \texttt{multi})
         should be raised to indicate this status.
      \item If F7 is encountered while \texttt{multi} is set,
         it is a continuation.
      \item If this continuation ends with F7, it is the last packet and
         \texttt{multi} should be cleared.
      \item If F7 is encountered while \texttt{multi} is \textsl{not} set,
         then the event is an escape sequence.
   \end{itemize}

\subsection{System Exclusive Sample Dump Standard}
\label{subsec:midi_meta_sysex_sample_dump_standard}

   This standard is not yet implemented in \textsl{Rtl66}.
   It provides for for ampler data dumps.
   Five of the basic messages are generic handshaking messages (ACK, NAK, Wait,
   Cancel \& EOF), are also used in other applications e.g. File Dump.
   The remaining messages are Dump Request, Dump Header, Data Packets, and a
   Sample Dump Extensions message.  The data formats are given in hexadecimal.

   Here we only summarize them.

\paragraph{Generic Handshaking Messages}
\label{paragraph:midi_seq66_generic_handshaking}

   In these descriptions, \texttt{pp} is the packet number.

   \begin{itemize}
      \item ACK:
         \texttt{F0 7E <device ID> 7F pp F7}.
         The first handshaking flag, meaning "Last data packet
         was received correctly. Send the next one."
         The packet number represents the packet being acknowledged as correct.
      \item NAK:
         \texttt{F0 7E <device ID> 7E pp F7}.
         The second handshaking flag, meaning "Last data packet was received
         incorrectly. Please resend."
         The packet number represents the packet being rejected.
      \item CANCEL:
         \texttt{F0 7E <device ID> 7D pp F7}.
         The third handshaking flag. It means "Abort dump."
         The packet number represents the packet on which the
         abort takes place.
      \item WAIT:
         \texttt{F0 7E <device ID> 7C pp F7}.
         The fourth handshaking flag, meaning "Do not send any more packets
         until told to do otherwise." This is important for systems in
         which the receiver (such as a computer) may need to
         perform other operations (such as disk access)
         before receiving the remainder of the dump. An ACK will
         continue the dump while a Cancel will abort the dump.
      \item EOF:
         \texttt{F0 7E <device ID> 7B pp F7}.
         A new generic handshaking flag added for the File Dump
         extension.
   \end{itemize}

\paragraph{Data Messages}
\label{paragraph:midi_seq66_data_message}

   \begin{itemize}
      \item DUMP HEADER.
         \texttt{F0 7E <device ID> 01 ss ss ee ff ff ff
            gg gg gg hh hh hh ii ii ii jj F7}.
         \begin{itemize}
            \item ss = Sample number (LSB first)
            \item ee = Sample format (# of significant bits from 8-28)
            \item ff = Sample period (1/sample rate) in nanoseconds (LSB first)
            \item gg = Sample length in words (LSB first)
            \item hh = Sustain loop start point word number (LSB first)
            \item ii = Sustain loop end point word number (LSB first)
            \item jj = Loop type (00 = forward only, 01 = backward/forward,
               7F = Loop off)
         \end{itemize}
      \item DUMP REQUEST.
         \texttt{F0 7E <device ID> 03 ss ss F7}, where ss =
         Requested sample, LSB first.
         Upon receiving this message, the sampler should check to see if the
         requested sample number falls in a legal range.
         If it is, the requested sample becomes the current sound
         number and is dumped to the requesting master following the
         procedure outlined in the documentation.
         If it is not within a legal range, the message should be ignored.
      \item DATA PACKET.
         \texttt{F0 7E <device ID> 02 kk <120 bytes> ll F7}, where
         kk = Running packet count (0-127); and
         ll = Checksum (XOR of 7E <device ID> 02 kk <120 bytes>).
      The total size of a data packet is 127 bytes. This is to prevent MIDI
      input buffer overflow in machines that may want to receive an
      entire message before processing it. 128
      bytes, or 1/2 page of memory, is
      considered the smallest reasonable buffer for modern MIDI instruments.
   \end{itemize}

   There are some more scenarios in the MIDI 1.0 detailed specification
   document, including:

   \begin{itemize}
      \item Sample dump extensions and a description of the sample dump
         scenario, too detailed to include here.
      \item Device inquiry and response.
         Messages used for device identification, categorized non-realtime
         SysEx general information messages.
      \item File dump provides a protocol for transmitting files from one
         computer to another using MIDI.
         Two primary motivations for this protocol: transmitting MIDI Files
         (and tempo maps) between computers and small microcomputer “boxes”;
         and transmitting files of \textsl{any} type between two computers
         of different types. The filename is sent with the file.
         The protocol includes a request, header, data packet, hand-shaking
         flags,
      \item MIDI Tuning.
   \end{itemize}

   There is so much more to look into.  Sigh.

\subsubsection{Universal SysEx Events (0xF0 0x7E and 0xF0 0x7F)}
\label{subsubsec:midi_meta_universalsysex_event}

   As noted earlier, 0x7E and 0x7F provide for non-realtime and realtime SysEx
   message.
   Immediately following these value is the "channel", which can be a
   manufacturer's ID or any value ranging from 0x00 to 0x7F (which is a
   wild-card for "all devices").

   \begin{verbatim}
      F7 7E channel subid1 subid2 data bytes F7      (non-realtime)
      F7 7F channel subid1 subid2 data bytes F7      (realtime)
   \end{verbatim}

   If \texttt{channel} is a device ID.
   If its value is F7, this is a global broadcast that all devices
   should heed.
   The \texttt{subid1} and \texttt{subid2}
   could be something like "01 00" for a long-form
   (full frame) time-code message.

%  F0 7E id 01                      Channel Pressure (Aftertouch)
%  F0 7E id 01                      or Sample Dump Header???
%  F0 7E id 02                      Polyphonic Key Pressure (Aftertouch)
%  F0 7E id 02                      or Sample Data Packet???
%  F0 7E id 03                      Controller (Control Change)
%  F0 7E id 03                      or Sample Dump Request???
%  F0 7E id 09 nn F7                GM System Enable / Disable (01 / 00)
%  F0 7F 00                         ???
%  F0 7F id 01 01 hh mm ss ff F7    MIDI Full Frame
%  F0 7F id 04 01 mm nn F7          MIDI Master Volume (vs Controller Volume)

   A detailed description is beyond the scope of this document.
   Some messages supported by these messages are the
   \textsl{MIDI Master Volume},
   \textsl{MIDI Full Frame},
   and the \textsl{General MIDI System Enable/Disable}
   messages.

   An interesting subset of these messages is \textsl{MIDI Show Control}
   (MSC):

   \begin{verbatim}
      F7 7F deviceid 02 commandformat command data F7      (MSC)
   \end{verbatim}

   See \cite{msc}.
   Some simple controls can be included in the 'ctrl' file's set of macros.

\paragraph{Rtl66 SysEx Handling}
\label{paragraph:midi_seq66_sysex_handling}

   \textsl{Rtl66} is slowly gaining support for reading, storing, and
   sending SysEx events within a sequence.
   \textsl{Rtl66} warns if the
   terminating 0xF7 SysEx terminator is not found at the expected length.
   Also, some malformed SysEx events have been encountered, and those are
   detected and handled as well.

   The format of a bare (i.e. not encoded in a MIDI file) SysEx message is
   like the following, complex case:

   \begin{verbatim}
      F0 manid devid modelid direction address   data checksum terminator
      F0 0x41  0x10  0x42    0x12      0x40007F  0x00 0x41     0xF7
   \end{verbatim}

   The "manid" is a manufacturer's identifier. 0x41 is Roland, 0x24 is
   Hohner. 0x00 is used for adding two more codes to greatly expand the ID
   list. 0x7E is used to denote a non-realtime message, and 0x7F denotes
   a realtime message.
   The "devid" is the device identifier (e.g. 0x10 for Roland devices).
   The "devid" indicates which devices will accept the SysEx message.
   The "modelid" is the model identifier (0x42 for most GS synths).
   The "direction" is 0x12 for sending information or 0x11 for requesting
   information.
   The "address" is a 3-byte value on which the SysEx message should act.
   Devices provide an address map to define what is at each address.
   For example, this address might define a "GS Reset", which performs
   an initialization.
   The "data" is either the data to send (one or more bytes), or the
   size of the data we are requesting.
   Some devices will include a "checksum" for data integrity.
   All SysEx messages end with an F7 byte.
   But see \sectionref{subsubsec:midi_meta_sysex_event}.



%-------------------------------------------------------------------------------
% vim: ts=3 sw=3 et ft=tex
%-------------------------------------------------------------------------------
